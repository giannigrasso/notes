\documentclass[../main.tex]{subfiles}
\begin{document}
\subsection{Metodo di bisezione}
Il metodo di bisezione è un metodo di ricerca incrementale in cui un intervallo che contiene uno zero della funzione viene ripetutamente 
dimezzato per localizzare con maggior precisone il valore esatto.

Considerando quindi una funzione $f: \mathbb{R} \longrightarrow \mathbb{R}$ che sia \underline{continua} possiamo eseguire i seguenti 
passaggi:
\begin{itemize}
    \item Scegliere due numeri $a$ e $b$ tali che \begin{align*}
        f(a) \cdot f(b) < 0
    \end{align*}
    \item Approssimare lo zero della funzione con il punto medio dell'intervallo $[a, b]$ \begin{align*}
        c = \frac{a + b}{2}
    \end{align*}
    \item Determinare in quale intervallo si trova il risultato valutando $f(a) \cdot f(c)$
    \begin{itemize}
        \item Se $f(a) \cdot f(c) < 0$: impostare $b = c$ e ripetere
        \item Se $f(a) \cdot f(c) > 0$: impostare $a = c$ e ripetere
        \item Se $f(a) \cdot f(c) = 0$: terminare il processo (lo zero è esattamente $c$)
    \end{itemize}
\end{itemize}

\vspace{1cm}
\subsubsection{Tolleranza}
Il secondo e il terzo passaggio sono da ripetere finchè non si trova lo zero esatto oppure, scenario più probabile, fino a che $f(c)$
sarà un valore sufficentemente vicino a zero secondo i nostri criteri.

Possiamo infatti definire una certa soglia di tolleranza sotto la quale consideriamo il risultato esatto, in questo caso procediamo come segue:
\begin{itemize}
    \item Definiamo un valore $\epsilon = 0.00...01$
    \item Controlliamo ad ogni iterazione che \begin{align*}
        \left\lvert f(c) \right\rvert < \epsilon
    \end{align*}
    se questa condizione è soddisfatta ci fermiamo, altrimenti ripetiamo il procedimento
\end{itemize}

Possiamo inoltre definire il numero di iterazioni che l'algoritmo impiegherà per raggiungere tale precisione:
\begin{align*}
    \frac{\left\lvert b - a\right\rvert }{2^n} \leq& \epsilon \\
    \Longleftrightarrow n \geq& \frac{log(b-a) - log(\epsilon)}{log(2)}
\end{align*}

\end{document}