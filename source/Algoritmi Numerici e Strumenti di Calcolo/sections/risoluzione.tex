\documentclass[../main.tex]{subfiles}
\begin{document}
In questa sezione verranno trattati i \underline{metodi di bracketing}, dei metodi che, dati due intervalli di una funzione,
trovano una soluzione approssimata di tale funzione.

\vspace{0.5cm}
\subsection{Metodo di bisezione}
Il metodo di bisezione è un metodo di ricerca incrementale in cui un intervallo che contiene uno zero della funzione viene ripetutamente 
dimezzato per localizzare con maggior precisone il valore esatto.

Considerando quindi una funzione $f: \mathbb{R} \longrightarrow \mathbb{R}$ che sia \underline{continua} possiamo eseguire i seguenti 
passaggi:
\begin{itemize}
    \item Scegliere due numeri $a$ e $b$ tali che \begin{align*}
        f(a) \cdot f(b) < 0
    \end{align*}
    \item Approssimare lo zero della funzione con il punto medio dell'intervallo $[a, b]$ \begin{align*}
        c = \frac{a + b}{2}
    \end{align*}
    \item Determinare in quale intervallo si trova il risultato valutando $f(a) \cdot f(c)$
    \begin{itemize}
        \item Se $f(a) \cdot f(c) < 0$: impostare $b = c$ e ripetere
        \item Se $f(a) \cdot f(c) > 0$: impostare $a = c$ e ripetere
        \item Se $f(a) \cdot f(c) = 0$: terminare il processo (lo zero è esattamente $c$)
    \end{itemize}
\end{itemize}

\vspace{1cm}
\subsubsection{Tolleranza}
Il secondo e il terzo passaggio sono da ripetere finchè non si trova lo zero esatto oppure, scenario più probabile, fino a che $f(c)$
sarà un valore sufficentemente vicino a zero secondo i nostri criteri.

Possiamo infatti definire una certa soglia di tolleranza sotto la quale consideriamo il risultato esatto, in questo caso procediamo come segue:
\begin{itemize}
    \item Definiamo un valore $\epsilon = 0.00...01$
    \item Controlliamo ad ogni iterazione che \begin{align*}
        \left\lvert f(c) \right\rvert < \epsilon
    \end{align*}
    se questa condizione è soddisfatta ci fermiamo, altrimenti ripetiamo il procedimento
\end{itemize}

Possiamo inoltre definire il numero di iterazioni che l'algoritmo impiegherà per raggiungere tale precisione:
\begin{align*}
    \frac{\left\lvert b - a\right\rvert }{2^{n+1}} \leq& \epsilon \\
    \Longleftrightarrow n \geq& \frac{log(b-a) - log(\epsilon)}{log(2)} - 1
\end{align*}

\pagebreak
\subsection{Regula falsi}
La regula falsi (o False Position) è un metodo alternativo al metodo di bisezione per trovare lo zero di una funzione. Utilizzando questo metodo
non viene fatta la media dei punti $a$ e $b$ per trovare $c$, viene traccaita una retta tra i due punti e $c$ diventa il punto che interseca l'asse
delle $x$ su quella retta. Questo ci fa risparmiare tempo e operazioni in determinati casi.
\vspace{0.5cm}
\begin{itemize}
    \item Scegliere due numeri $a$ e $b$ tali che \begin{align*}
        f(a) \cdot f(b) < 0
    \end{align*}
    \item Calcoliamo c come punto della retta $ab$ che interseca l'asse $x$\begin{align*}
        c = a-\frac{f(a)}{\frac{f(b)-f(a)}{(b-a)}}
    \end{align*}
    \item Determinare in quale intervallo si trova il risultato valutando $f(a) \cdot f(c)$
    \begin{itemize}
        \item Se $f(a) \cdot f(c) < 0$: impostare $b = c$ e ripetere
        \item Se $f(a) \cdot f(c) > 0$: impostare $a = c$ e ripetere
        \item Se $f(a) \cdot f(c) = 0$: terminare il processo (lo zero è esattamente $c$)
    \end{itemize}
\end{itemize}
\vspace{0.5cm}
Ci sono alcuni casi però in cui questa metodo risulta sconveniente rispetto al metodo di bisezione. Per risolvere questo problema possiamo
combinare i due metodi, si parte quindi con un iterazione di regula falsi seguito da una o più interazioni del metodo di bisezione.
Per determinare quando passare di nuovo al metodo di bisezione guardiamo dove l'intervallo rimasto uguale durante la prima iterazione cambia.

\end{document}