\documentclass[../main.tex]{subfiles}
\begin{document}
\subsection{Algoritmo di Horner}
\subsubsection{Valutazione di un polinomio}
Per capire Horner dobbiamo prima fare degli step intermedi che dimostrano la sua effettiva utilità, consideriamo un generico polinomio di terzo grado:
$$
    P(x) = a_3x^3 + a_2x^2 + a_1x + a_0
$$
dove $a_0, ..., a_3$ sono coefficienti reali qualsiasi.

Calcoliamo quante operazioni sono necessarie per calcolare il valore del polinomio $P(x)$ in un generico punto $x_0$ effettuando le moltiplicazioni e le addizioni richieste:
$$
    P(x_0) = \underbrace{a_3x_0^3}_{\text{3 molt.}} + \underbrace{a_2x_0^2}_{\text{2 molt.}} + \underbrace{a_1x_0}_{\text{1 molt.}} + a_0
$$
notiamo che sono necessarie esattamente 6 moltiplicazioni e 3 addizioni.

Ora invece consideriamo un generico polinomio di grado $n$:
$$
    P(x) = \sum_{k=0}^{n} a_kx^k = a_0 + a_1x + ... + a_nx^n
$$
dove $a_0, ..., a_n$ sono coefficienti reali qualsiasi.

Abbiamo quindi lo stesso scenario del caso precedente ma con un polinomio di grado $n$ invece che di grado $3$. Procediamo in modo analogo a quanto fatto prima per determinare il numero di operazioni necessarie per calcolare il valore del polinomio $x_0$:
$$
    P(x_0) = \underbrace{a_nx_0^n}_{\text{n molt.}} + \underbrace{a_{n-1}x_0^{n-1}}_{\text{n - 1 molt.}} + ... + \underbrace{a_1x_0}_{\text{1 molt.}} + a_0
$$
Sfruttando la formula introdotta nello scorso capitolo risulta che servono:
$$
    \underbrace{\frac{n(n+1)}{2} + n}_{\text{molt.}} = \frac{1}{2}n^2 + \frac{3}{2}n
$$
operazioni, che corrispondono a una complessità temporale \textbf{quadratica}, ossia $O(n^2)$, poiché il termine maggiore è $n^2$.

Se ora consideriamo nuovamente il polinomio di terzo grado di prima possiamo notare che esso può essere riscritto come:
$$
    P(x) = a_3x^3 + a_2x^2 + a_1x + a_0 = ((a_3x + a_2)x + a_1)x + a_0
$$
la cui correttezza può essere verificata sviluppando i conti. Notiamo che il numero di operazioni è diminuito, prima erano 6 moltiplicazioni e 3 addizioni mentre con la nuova formula le moltiplicazioni sono 3.

Questa formula può essere generalizzata ad un polinomio di grado $n$:
$$
    P(x_0) = (((a_nx_0 + a_{n-1})x_0 + a_{n-2}) \cdot ...)x_0 + a_0
$$
che corrisponde all'\textbf{algoritmo di Horner}, che consente di risolvere un polinomio mediante $n$ moltiplicazioni e $n$ addizioni (come visto prima). In totale sono quindi necessarie $2n$ operazioni e ciò corrisponde a una complessita temporale \textbf{lineare}, ossia $O(n)$.

\pagebreak
\subsubsection{Applicazione}
L'algoritmo di Horner è utile per passare da un sistema di numerazione qualsiasi al sistema decimale (ad esempio da binario a decimale).

Se consideriamo ad esempio il numero $3152$, possiamo scriverlo come:
\begin{align*}
    3152  &= 3 \cdot 1000 + 1 \cdot 100 + 5 \cdot 10 + 2 \\
    &= 3 \cdot 10^3 + 1 \cdot 10^2 + 5 \cdot 10^1 + 2 \cdot 10^0
\end{align*}
A ciascuna cifra viene attribuita una diversa potenza di 10 in base alla posizione che occupa all'interno del numero considerato. Un generico numero $x \in \mathbb{N}$ può essere scritto come:
$$
    x = a_n10^n + a_{n-1}10^{n-1} + ... + a_0 = \sum_{i=0}^{n} a_i10^i
$$
dove $a_0,...,a_n$ corrisponde alle cifre del numero.

Possiamo notare che la scrittura riportata è analoga a quella vista nella dimostrazione, che possiamo quindi riscrivere il numero come:
$$
    x = (((a_n \cdot 10 + a_{n-1}) \cdot 10 + a_{n-2}) \cdot ...) \cdot 10 + a_0
$$

Per quanto riguarda gli altri sistemi numerici vale lo stesso discorso, consideriamo quindi un numero in una qualsiasi base $b$ e scriviamolo nel corrispondente in base 10:
$$
    x = a_nb^n + a_{n-1}b^{n-1} + ... + a_0 = \sum_{i=0}^{n} a_ib^i
$$
Ad esempio se vogliamo convertire $1201_3$ in base 10, svolgiamo questi calcoli:
\begin{align*}
    1201_3 &= \sum_{i=0}^{3} a_i3^i \\
    &=1 \cdot 3^3 + 2 \cdot 3^2 + 0 \cdot 3^1 + 1 \cdot 3^0 \\
    &=27 + 18 + 1 = 46_{10}
\end{align*}

Che utilizzando l'algoritmo di Horner diventa:
\begin{align*}
    1201_3 &= ((1 \cdot 3 + 2) \cdot 3 + 0) \cdot 3 + 1 \\
    &= (5 \cdot 3 + 0) \cdot 3 + 1 \\
    &= 15 \cdot 3 + 1 = 46_{10}
\end{align*}

\subsubsection{Horner inverso}
L'algoritmo di Horner inverso, o \textbf{algoritmo del modulo}, consente invece di passare da base 10 a base $b$.

\pagebreak
\subsection{Complemento a due}
Vogliamo ora ampliare i numeri rappresentabili, grazie all'\textbf{algoritmo di Horner} infatti possiamo rappresentare solo i numeri
naturali $\mathbb{N} = \{ 0, 1, 2, 3, 4, ... \} $. Ci serve quindi un modo per rappresentare anche i numeri interi
$\mathbb{Z} = \{ ..., -3, -2, -1, 0, 1, 2, 3, ... \} $, ovvero il complemento a due. Gli insiemi interi positivi e negativi sono indicati
rispettivamente con $\mathbb{Z^+}, \mathbb{Z^-}$.

Il complemento a due è in grado di rappresentare i numeri interi tramite l'uso delle seguenti regole:
\begin{enumerate}
    \item il primo bit indica sempre il segno del numero (\textbf{0 positivo} e \textbf{1 negativo})
    \item se il primo bit è 0, il numero viene letto in base 10 usando $Horner(bit)$
    \item se il primo bit è 1, il numero viene letto in base 10 come $-(2^n - Horner(bit))$
\end{enumerate}

Facciamo alcuni esempi per quanto riguarda l'ultimo caso:
\begin{align*}
    &1100_2 \text{, il primo bit è 1, segno negativo}\\
    &Horner(1100) = (1 \cdot 2 + 1)\cdot 2 + ... = 12 \\
    &-(2^4 - Horner(1100)) = -(16-12) = -4_{10}
\end{align*}
un altro esempio:
\begin{align*}
    &1011 = 11_{10}\\
    &-(2^4 - 11) = -(16-11) = -5
\end{align*}

Consideriamo ora anche l'\textbf{algoritmo di Horner inverso}, che ci permette di passare da un numero intero alla sua rappresentazione
in binario. Abbiamo quindi due casi distinti:
\begin{enumerate}
    \item se il numero $x \in \mathbb{Z}_0^+$, applichiamo l'algoritmo inverso
    \item se il numero $x \in \mathbb{Z}_0^-$, dobbiamo risolvere la seguente equazione per i \underline{bit}: \begin{align*}
        &x = -(2^n - Horner(bit)) \\
        \Longleftrightarrow &x = -2^n + Horner(bit) \\
        \Longleftrightarrow &Horner(bit) = 2^n + x \\
        \Longleftrightarrow &bit = invHorner(2^n+x)
    \end{align*}
\end{enumerate}
dove $invHorner$ indica l'inverso dell'algoritmo di Horner e $n$ il numero di bit da cui è composta la memoria.

\subsubsection{Interpretare le consegne}
Durante lo svolgimento degli esercizi verranno menzionati spesso \code{short int} e \code{unsigned short int} o simili. Possiamo 
descriverli in questo modo:
\begin{itemize}
    \item \textbf{Complemento a due}
    \begin{itemize}
        \item short int 16 bit
        \item int 32 bit
        \item long int 64 bit
    \end{itemize}
    \item \textbf{Unsigned (come numero naturale)}
    \begin{itemize}
        \item short int 16 bit
        \item int 32 bit
        \item long int 64 bit
    \end{itemize}
\end{itemize}

\subsection{Horner su numeri razionali}
Vogliamo ampliare ulteriormente i numeri rappresentabili. Fino ad ora pssiamo rappresentare tutti i numeri interi 
$\mathbb{Z}={...,-3,-2,-1,0,+1,+2,+3,...}$. Ci serve quindi un modo per rappresentare i numeri razionali, per farlo possiamo utilzare 
\textbf{Horner} sulla parte decimale di un numero.

Per convertire numeri razionali da base $10$ a base $b$ possiamo usare sempre l'algoritmo di Horner, con alcuni accorgimenti:
\begin{enumerate}
    \item separiamo la parte intera dalla parte decimale
    \item per la parte intera, usiamo l'algoritmo di Horner normalmente
    \item per la parte decimale, usiamo una versione modificata che utilizza la funzione \code{INT()} per calcolare la parte intera di un numero
\end{enumerate}

Ecco un esempio da base $10$ a base \color{blue}$2$\color{black}:
\begin{align*}
    int(0.1875 * \color{blue}2\color{black}) =& 0 (MSB)\\
    int(0.375 * \color{blue}2\color{black}) =& 0\\
    int(0.75 * \color{blue}2\color{black}) =& 1\\
    int(0.5 * \color{blue}2\color{black}) =& 1 (LSB) \\
    \Rightarrow 0.1875_{10} =& 0.0011_2
\end{align*}

e uno da base 2 a base 10, in questo caso si parte dal \textit{LSB}:
\begin{align*}
    \color{blue}1\color{black} \longrightarrow \frac{0}{2} + \color{blue}1\color{black} =& \color{red}1\\
    \color{green}1\color{black} \longrightarrow \frac{\color{red}1\color{black}}{2} + \color{green}1\color{green} =& \color{orange}1.5\color{black}\\
    \color{magenta}0\color{black} \longrightarrow \frac{\color{orange}1.5\color{black}}{2} + \color{magenta}0\color{black} =& \color{violet}0.75\color{black}\\
    \color{cyan}0\color{black} \longrightarrow \frac{\color{violet}0.75\color{black}}{2} + \color{cyan}0\color{black} =& 0.375\\
    \longrightarrow& \frac{0.375}{2} = 0.1875\\
\end{align*}

\pagebreak
\subsection{Horner su numeri reali}
Al punto in cui siamo arrivati ora non abbiamo modo di esprimere delle radici con Horner. Vogliamo dunque ampliare l'insieme razionale
$\mathbb{Q}$ ed estenderlo a tutti i numeri reali $\mathbb{R}$.

\vspace{0.5cm}
Per evitare errori di approssimazioni utilizziamo la notazione scientifica, consideriamo però una notazione scientifica in 
base $2$ e non in base $10$ come siamo abituati.

Per farlo possiamo usare il formato float, dati $32$ bit esso si basa sulla seguente allocazione:
\begin{enumerate}
    \item il primo bit per il segno ($1$ se negativo, $0$ se positivo)
    \item $8$ bit per l'esponente
    \item i rimanenti $23$ bit per la mantissa (1.XXX)
    \item infine la base (della notazione scientifica) è sempre pari a $2$
\end{enumerate}
\textbf{Nota:} è \underline{sempre} possibile scrivere il numero da rappresentare nella forma della mantissa.

\textbf{Nota:} il formato float \underline{non} utilizza il complemento a due.

\vspace{0.5cm}
%\subsubsection{Algoritmo inverso}
Con la notazione appena fornita non possiamo rappresentare lo zero, i numeri più vicini allo zero che possiamo rappresentare sono
\begin{align*}
    + 1.0 * 2^{-127} =& -\frac{1}{127}\\
    - 1.0 * 2^{-127} =& \phantom{-}\frac{1}{127} 
\end{align*}
entrambi \underline{diversi da zero}.

\vspace{0.5cm}
\subsection{Applicazione}
\begin{itemize}
    \item Esponente, $exp = horner(bit) -127$
    \item Numero, $n = sgn * 1.XXX * 2^{exp}$
\end{itemize}

\subsubsection{Numeri macchina}
C'è però un problema, i numeri reali sono infiniti, mentre i numeri rappresentabili da un computer sono finiti. Dobbiamo quindi scegliere
dei \underline{numeri macchina}, ovvero i valori che possiamo effettivamente rappresentare con 23 bit, i valori che non sono numeri macchina vengono 
aapprossimati al numero macchina più vicino. 

Se ad esempio un numero intero non è rappresentabile, troveremo il numero macchina più vicino ad esso e, sottraendolo al numero di partenza
troveremo l'errore assoluto di quella rappresentazione.
\begin{itemize}
    \item Errore assoluto
    \begin{itemize}
        \item $x$: numero reale
        \item $x_m$: numero macchina \begin{align*}
            e_a =& \left\lvert x - x_m\right\rvert  \\
            e_a <& 2^{p-s-1}
        \end{align*}
        dove nell'ultima formula $p$ è l'esponente del numero e $s$ il numero di bit della mantissa.
    \end{itemize}
    \item Errore relativo \begin{align*}
        e_r =& \frac{e_a}{x} \\
        e_r =& \frac{\left\lvert x - x_m\right\rvert}{x}
    \end{align*}
\end{itemize}

\end{document}