\documentclass{article}
\usepackage{hyperref}
\usepackage[italian]{babel} % italian language
\usepackage{fullpage} % margins
\usepackage{listings} % code formatting
\usepackage{subfiles}       % subfiles
\usepackage{graphicx}       % images
\usepackage[dvipsnames]{xcolor} % additional colors
\usepackage{epigraph}
\usepackage{supsi}
\usepackage{mdframed}
\usepackage{csquotes}
\usepackage{parskip}        % skip line
\usepackage{amsfonts}       % math symbols
\usepackage{amssymb}        % math symbols
\usepackage{amsmath}        % other math stuff
\usepackage{tikz}           % math graphs
\usepackage{enumerate}      % for particolar enumerations


\title{
    Fisica \\
    \phantom{}\\
    \large SUPSI Dipartimento Tecnologie Innovative
}
\author{Gianni Grasso}
\supsisetup{Fisica}

\begin{document}
\maketitle
\hphantom{ }
\vspace{14.5cm}

\textbf{Classe}: I1B

\textbf{Anno scolastico}: 2024/2025
\pagebreak


\tableofcontents
\pagebreak

\section{Introduzione}
\subfile{sections/introduzione}
\pagebreak

\section{Cinematica unidimensionale}
\subfile{sections/cinematica}
\pagebreak

\section{Vettori}
\subfile{sections/vettori}
\pagebreak

\section{Cinematica bidimensionale}
\subfile{sections/cinematicaBidimensionale}
\pagebreak


\end{document}