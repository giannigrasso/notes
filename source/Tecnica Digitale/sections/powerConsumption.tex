\documentclass[../main.tex]{subfiles}
\begin{document}
La Potenza viene espressa in \textit{Watt} e il suo consumo si divide in:
\begin{itemize}
    \item Consumo dinamico
    \item Consumo statico
\end{itemize}

Il consumo dinamico rappresenta il consumo durante l'utilizzo di un dispositivo e vale:
$$
    P_{dynamic} = \frac{1}{2} \cdot C \cdot V_{DD}^2 \cdot f
$$
Dove $C$ rappresenta la capacità, $V_{DD}$ la tensione e $f$ la frequenza. Quest'ultima inoltre
è il paramentro più facile da controllare quando vogliamo alzare o abbassare il livello di consumo dinamico.

Per quanto riguarda il consumo statico, esso rappresenta il consumo "a riposo" ed è determinato come:
$$
    P_{static} = I_{DD} \cdot V_{DD}
$$

E ovviamente il consumo totale è calcolato dalla somma degli altri due:
\begin{align*}
    P_{total} =& P_{dynamic} + P_{static} \\
    =&\frac{1}{2} \cdot C \cdot V_{DD}^2 \cdot f \phantom{-} + \phantom{-} I_{DD} \cdot V_{DD}
\end{align*}
    

\end{document}