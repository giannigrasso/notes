\documentclass{article}
\usepackage{hyperref}
\usepackage[italian]{babel} % italian language
\usepackage{fullpage} % margins
\usepackage{listings} % code formatting
\usepackage{subfiles}       % subfiles
\usepackage{graphicx}       % images
\usepackage{xcolor}         % colors
\usepackage{epigraph}
\usepackage{supsi}
\usepackage{mdframed}
\usepackage{csquotes}
\usepackage{parskip}        % skip line
\usepackage{amsfonts}       % math symbols
\usepackage{amssymb}        % math symbols
\usepackage{amsmath}        % other math stuff
\usepackage{tikz}           % math graphs

\title{
    Ambienti Operativi \\
    \phantom{}\\
    \large SUPSI Dipartimento Tecnologie Innovative
}
\author{Gianni Grasso}
\supsisetup{Ambienti Operativi}

\begin{document}
\maketitle
\hphantom{ }
\vspace{14.5cm}

\textbf{Classe}: I1B

\textbf{Anno scolastico}: 2024/2025
\pagebreak


\tableofcontents
\pagebreak

\section{Introduzione}
\subfile{sections/introduzione}
\pagebreak


\section{Bash}
\subfile{sections/bash}
\pagebreak

\section{Scripting}
\subfile{sections/scripting}
\pagebreak

\section{Espressioni regolari}
\subfile{sections/regexp}
\pagebreak

\end{document}
