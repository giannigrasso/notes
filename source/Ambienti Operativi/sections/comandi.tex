\documentclass[../main.tex]{subfiles}
\begin{document}
Ecco una lista di comandi utili visti durante il corso:
\begin{itemize}
    \item \code{history} \\
    Stampa la cronologia dei comandi
    \begin{lstlisting}[style=bash]
        history
    \end{lstlisting}
    \code{ctrl + r} per cercare
    \item \code{touch} \\
    Aggiorna la data dell'ultima modifica del file, se il file non esiste ne crea uno
    \begin{lstlisting}[style=bash]
        touch filename
    \end{lstlisting}
    \item \code{rm} \\
    Cancella un file o una directory
    \begin{lstlisting}[style=bash]
        rm filename #cancella un file

        rmdir directory #cancella una directory vuota

        rm -r #cancella una directory e tutti i file al suo interno
    \end{lstlisting}
    \item \code{mv} e \code{cp} \\
    Rinomina, copia o sposta un file
    \begin{lstlisting}[style=bash]
        mv filename newFilename #rinomina un file

        cp filename copyFilename #copia un file
        cp -r directory copyDirectory #copia una cartella

        mv filename directory #sposta un file
    \end{lstlisting}
    \item \code{cat} \\
    Visualizza il contenuto di un file
    \begin{lstlisting}[style=bash]
        cat filename
    \end{lstlisting}
    \item \code{seq} \\
    Stampa una sequenza di numeri
    \begin{lstlisting}[style=bash]
        # inizia da 1 e finisce a 100 (inrementa di 1 di default).
        seq 1 100 > file.txt

        cat file.txt
        # 1
        # 2
        # ...
        # 100
    \end{lstlisting}
    \item \code{find} \\
    Cerca dei file secondo i criteri imposti. Argomenti comuni
    \begin{itemize}
        \item \code{-name}: il nome dei file
        \item \code{-iname}: il nome dei file (non case sensitive)
        \item \code{-user}: il proprietario dei file
        \item \code{-size}: la dimensione dei file
        \item \code{-exec}: esegue un comando per ogni file trovato. \code{-exec rm \{\} \textbackslash;}
    \end{itemize}
    \begin{lstlisting}[style=bash]
        # trova i file che finiscono con .jpg e li copia nella cartella /destinazione
        find / -name '*.jpg' -exec copy {} /destination \;
    \end{lstlisting}
    \item \code{head} \\
    Stampa le prime $n$ righe di un file
    \begin{lstlisting}[style=bash]
        head -n 50 filename #stampa le prime 50 righe del file

        cat filename | head -n 1 #stampa la prima riga del file

        cat filename | head -n -3 #stampa tutto il contenuto del file tranne le prime 3 righe
    \end{lstlisting}
    \item \code{tail} \\
    Stampa le ultime $n$ righe di un file
    \begin{lstlisting}[style=bash]
        tail -n 50 filename #stampa le ultime 50 righe del file

        cat filename | tail -n 1 #stampa l'ultima riga del file
    \end{lstlisting}
    \item \code{wc} \\
    Conta le parole passate come input
    \begin{lstlisting}[style=bash]
        # stampa il numero di linee di file.txt
        cat file.txt | wc -l
    \end{lstlisting}
    \item \code{tr} \\
    Sostituisce una parola con un altra
    \begin{lstlisting}[style=bash]
        # sostituisce tutte le lettere `a` con `b`
        cat file.txt | tr a b
    \end{lstlisting}
    \item \code{cut} \\
    Estrae porzioni di testo, utilizzando delimitatori o posizioni specifiche.
    \begin{lstlisting}[style=bash]
        # estrae il secondo campo di testo dividendo il file a ogni `;`
        cat file.txt | cut -d ";" -f2
    \end{lstlisting}
    \item \code{grep} \\
    Filtra le righe di un file
    \begin{lstlisting}[style=bash]
        # Visualizza le righe di file.txt che contengono la parola `ciao`
        grep ciao file.txt
    \end{lstlisting}
    \item \code{tee} \\
    Redirige l'output
    \begin{lstlisting}[style=bash]
        find / -name "z*" 2>&1 | tee filename #cerca dei file e redirige l'output sia nel file che sullo schermo
    \end{lstlisting}
    \item \code{sort} \\
    Ordina l'output di un comando o visualizza il contenuto ordinato di un file
    \begin{lstlisting}[style=bash]
        cat filename | sort #ordine alfabetico

        cat filename | sort -r #ordine alfabetico al contrario
    \end{lstlisting}
    \item \code{uniq} \\
    Stampa a schermo le righe senza duplicati (se le righe sono adiacenti)
    \begin{lstlisting}[style=bash]
        cat filename | sort | uniq
    \end{lstlisting}
    \item \code{groups} \\
    Visualizza a quali gruppi appartengo
    \begin{lstlisting}[style=bash]
        groups
    \end{lstlisting}
    \item \code{newgrp} \\
    Cambio il mio gruppo corrente
    \begin{lstlisting}[style=bash]
        newgrp adm
    \end{lstlisting}
    \item \code{chmod} \\
    Cambia i permessi di un file
    \begin{lstlisting}[style=bash]
        chmod u=rw,g=w,o=- filename

        chmod ugo=rw filename
    \end{lstlisting}
    \item \code{chown} e \code{chgrp} \\
    Cambia gruppo o proprietario di un file
    \begin{lstlisting}[style=bash]
        chown user filename

        chgrp group filename
    \end{lstlisting}
    \item \code{parallel} \\
    Esegue processi in parallelo
    \begin{lstlisting}[style=bash]
        cat filename | parallel rm {} #elimina ogni file che ha il nome uguale ad una riga del file "filename"

        cat filename | parallel -n 2 echo "Riga" #-n 2 prende due righe alla volta

        cat filename | parallel --pipe -n2 wc -l
    \end{lstlisting}
\end{itemize}
\pagebreak

\end{document}