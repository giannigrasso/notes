\documentclass[../main.tex]{subfiles}
\begin{document}
Ecco una lista di comandi utili visti durante il corso:
\begin{itemize}
    \item \code{history} \\
    Stampa la cronologia dei comandi
    \begin{lstlisting}[style=bash]
        history
    \end{lstlisting}
    \code{ctrl + r} per cercare
    \item \code{touch} \\
    Aggiorna la data dell'ultima modifica del file, se il file non esiste ne crea uno
    \begin{lstlisting}[style=bash]
        touch filename
    \end{lstlisting}
    \item \code{seq} \\
    Stampa una sequenza di numeri
    \begin{lstlisting}[style=bash]
        # inizia da 1 e finisce a 100 (inrementa di 1 di default).
        seq 1 100 > file.txt

        cat file.txt
        # 1
        # 2
        # ...
        # 100
    \end{lstlisting}
    \item \code{find} \\
    Cerca dei file secondo i criteri imposti. Argomenti comuni
    \begin{itemize}
        \item \code{-name}: il nome dei file
        \item \code{-iname}: il nome dei file (non case sensitive)
        \item \code{-user}: il proprietario dei file
        \item \code{-size}: la dimensione dei file
        \item \code{-exec}: esegue un comando per ogni file trovato. \code{-exec rm \{\} \textbackslash;}
    \end{itemize}
    \begin{lstlisting}[style=bash]
        # trova i file che finiscono con .jpg e li copia nella cartella /destinazione
        find / -name '*.jpg' -exec copy {} /destination \;
    \end{lstlisting}
    \item \code{wc} \\
    Conta le parole passate come input
    \begin{lstlisting}[style=bash]
        # stampa il numero di linee di file.txt
        cat file.txt | wc -l
    \end{lstlisting}
    \item \code{tr} \\
    Sostituisce una parola con un altra
    \begin{lstlisting}[style=bash]
        # sostituisce tutte le lettere `a` con `b`
        cat file.txt | tr a b
    \end{lstlisting}
    \item \code{cut} \\
    Estrae porzioni di testo, utilizzando delimitatori o posizioni specifiche.
    \begin{lstlisting}[style=bash]
        # estrae il secondo campo di testo dividendo il file a ogni `;`
        cat file.txt | cut -d ";" -f2
    \end{lstlisting}
    \item \code{grep} \\
    Filtra le righe di un file
    \begin{lstlisting}[style=bash]
        # Visualizza le righe di file.txt che contengono la parola `ciao`
        grep ciao file.txt
    \end{lstlisting}
\end{itemize}

\end{document}