\documentclass[../main.tex]{subfiles}
\begin{document}
La prima volta che apriremo un terminale potremo notare che il cursore è preceduto da: \textbf{utente@host:path\$}, dove utente sta per il nome dell'utente connesso alla macchina, host il nome del server e path il percorso corrente.

È importante ricordare che in bash è tutto \textbf{case-sensitive}, sia i comandi che i nomi dei file e delle directory.

\subsection{Comandi}
La struttura di un comando da terminale è definita in questo modo:
\begin{lstlisting}[language=bash]
comando [-OPTION]... [ARGUMENT]...
\end{lstlisting}
Dove OPTION indica le opzioni (-a, -b, ...) e ARGUMENT gli argomenti, che possono essere obbligatori oppure no a dipendenza del comando.

Per consultare il manuale di un comando digitare:
\begin{lstlisting}[language=bash]
man comando
\end{lstlisting}

\subsection{Globbin}
Il Globbin consiste nell'utilizzare un pattern con uno o più caratteri "wildcard" per trovare o fare azioni sui file.
\begin{itemize}
    \item *, corrispondenza con zero o più caratteri qualsiasi
    \item ?, corrispondenza con esattamente un carattere
    \item $\lbrack \phantom{-} \rbrack$, corrispondenza tra un gruppo di caratteri
    \item $\lbrack \text{a-z} \rbrack$, un carattere dalla 'a' alla 'z'
    \item $\lbrack$\^{}abc$\rbrack$, non-corrispondenza tra un gruppo di caratteri
\end{itemize}
\end{document}