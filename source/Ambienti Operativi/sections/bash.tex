\documentclass[../main.tex]{subfiles}
\begin{document}
La prima volta che apriremo un terminale potremo notare che il cursore è preceduto da: \code{utente@host:path\$}, dove utente sta per il nome dell'utente connesso alla macchina, host il nome del server e path il percorso corrente.

È importante ricordare che in bash è tutto \textbf{case-sensitive}, sia i comandi che i nomi dei file e delle directory.

\subsection{Comandi}
\subfile{comandi}

\subsection{Globbing}
\subfile{globbing}

\subsection{Redirezione}
\subfile{redirezione}

\subsection{Esecuzione sequenziale e condizionale}
\subfile{sequenzialeCondizionale}

\pagebreak
\subsection{Gestione dei permessi}
\subfile{permessi}

\pagebreak
\subsection{Elaborazione parallela}
\subfile{elaborazioneParallela}

\end{document}