\documentclass[../main.tex]{subfiles}
\begin{document}
Per visualizzare ed elencare i processi presenti sul sistema possiamo usare:
\begin{itemize}
    \item \code{ps}
    \item \code{top}
    \item \code{htop}
\end{itemize}

\vspace{1cm}
Le informazioni sulla cpu a disposizione sono presenti nel file \code{/proc/cpuinfo}.

\vspace{1cm}
Per vedere le informazioni relative al tempo di un comando possiamo usare il comando \code{time} prima dell'effettivo comando.

\vspace{1cm}
\subsubsection{Approccio manuale}
Con il comando \code{split} possiamo dividere un file in $n$ file in base al numero di linee che esso contiene. Ad esempio:
\begin{lstlisting}[style=bash]
    $ wc -l file.tsv
    9804311 file.tsv #Il numero di righe del file
    
    $ split -l 4902155 file.tsv #Separa il file ogni 4902155 righe (la metà), verranno creati due file
\end{lstlisting}

\subsubsection{GNU Parallel}
Il comando \code{parallel} ci permette di eseguire un comando per ogni riga di un file e lo fa in modo parallelo. Funziona in modo simile
al comando \code{exec} con \code{find}.

Il comando \code{parallel} riceve lo \code{stdin} e lo passa al comando successivo, è utile con comandi che funzionano con lo \code{stdin},
come ad esempio \code{rm} o \code{touch}. Facciamo qualche esempio:
\begin{lstlisting}[style=bash]
    cat lista | parallel rm {}

    cat lista | parallel echo {} "Riga"
\end{lstlisting}

\vspace{1cm}
Si può anche specificare quante righe alla volta usare come parametro, ad esempio:
\begin{lstlisting}[style=bash]
    cat lista | parallel -n 2 echo "Ciao {1} mondo {2}"
\end{lstlisting}

\vspace{1cm}
Infine, possiamo usare il comando \code{--pipe} per passare le righe tramite pipe. Questo è utile solo quando vogliamo eseguire operazioni
con più righe alla volta:
\begin{lstlisting}[style=bash]
    cat lista | parallel --pipe -n2 wc -l
\end{lstlisting}

\end{document}