\documentclass[../main.tex]{subfiles}
\begin{document}
Sui sistemi Unix i programmi hanno accesso a tre stream di input e output.
\begin{itemize}
    \item \textbf{stdin} (standard input) [0], input dalla tastiera
    \item \textbf{stdout} (standard output) [1], output su schermo
    \item \textbf{stderr} (standard error) [2], output degli errori su schermo
\end{itemize}

Possiamo redirezionare gli input e output di default. Per redirezionare l'input useremo il carattere '<', per l'output '\textbf{1>}' (sovrascrive)
o '\textbf{1>\phantom{}>}' (append) e per gli errori '\textbf{2>}'. Ad esempio:

\begin{lstlisting}[language=bash, frame=l]
    utente@host:~$ cat leggi.txt >> aggiungi.txt
\end{lstlisting}
Redireziona lo stdout del comando \textit{cat} sul file \textit{aggiungi.txt}.

Per redirezionare sia lo stdout che lo stderr si usa '\textbf{>\&}'. Inoltre è possibile redirezionare un qualsiasi output a \textbf{/dev/null} per non visualizzarlo.

\subsection{Pipe}
La pipe è un buffer che permette di passare lo stdout di un comando come stdin di un altro, da sinistra a destra. Il seguente comando 
ad esempio prende l'input del comando \textit{cat} e lo passa al comando \textit{grep}:
\begin{lstlisting}[language=bash, frame=l]
    utente@host:~$ cat A.txt| grep "ABC"
\end{lstlisting}

È possibile redirezionare lo stderr nello stdout, in questo caso possiamo usare '\textbf{2>\&1}':
\begin{lstlisting}[language=bash, frame=l]
    utente@host:~$ ls nonesistente 2>&1 >/dev/null | grep impossibile
\end{lstlisting}

Ecco una lista di comandi utili:
\begin{itemize}
    \item head, restituisce le prime $n$ rghe di un file
    \item tail, restituisce le ultime $n$ rghe di un file
    \item tr, sostituisce una lettera con un altra all'interno di un file
    \item cut, estrae i primi $n$ caratteri di ogni linea di un file
    \item grep, visualizza tutto ciò che contiene una determinata stringa
    \item sort, ordina il contenuto dell'output, di default in ordine alfabetico
    \item wc, conta il numero di righe di un output
    \item uniq, stampa a schermo le righe di un output senza duplicati (righe adiacenti uguali)
\end{itemize}

\end{document}