\documentclass[../main.tex]{subfiles}
\begin{document}
Sui sistemi Unix i programmi hanno accesso a tre stream di input e output.
\begin{itemize}
    \item \code{stdin} \code{[0]}, input dalla tastiera
    \begin{lstlisting}[style=bash]
        > cat > greetings.txt
    \end{lstlisting}
    \item \code{stdout} \code{[1]}, output su schermo
    \begin{lstlisting}[style=bash]
        cat file > /dev/null
    \end{lstlisting}
    \item \code{stderr} \code{[2]}, output degli errori su schermo
    \begin{lstlisting}[style=bash]
        cat nonexistingfile 2> errors.txt
    \end{lstlisting}
\end{itemize}
\textbf{Nota:} Di default \code{1>} non include gli errori e sovrascrive il file di destinazione. Per fare l'append usare \code{>>}.

È anche possibile redirezionare stdout e stderr insieme, tuttavia non è buona pratica farlo in un file come nell'esempio sottostante:
\begin{lstlisting}[style=bash]
    cat file nonexisting >& results.txt
\end{lstlisting}

\subsubsection{Pipe}
La pipe è un buffer che permette di passare lo stdout di un comando come stdin di un altro, da sinistra a destra. Il seguente comando 
ad esempio prende l'input del comando \code{cat} e lo passa al comando \code{grep}:
\begin{lstlisting}[style=bash]
    cat A.txt| grep "ABC"
\end{lstlisting}

È possibile redirezionare lo stderr \textbf{nello} stdout, in questo caso possiamo usare '\code{2>\&1}':
\begin{lstlisting}[style=bash]
    ls nonesistente 2>&1 >/dev/null | grep impossibile
\end{lstlisting}

\end{document}