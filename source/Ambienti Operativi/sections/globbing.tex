\documentclass[../main.tex]{subfiles}
\begin{document}
Il Globbin consiste nell'utilizzare un pattern con uno o più caratteri "wildcard" per trovare o fare azioni sui file.
\begin{itemize}
    \item \code{*}, corrispondenza con zero o più caratteri qualsiasi
    \item \code{?}, corrispondenza con esattamente un carattere
    \item \code{\lbrack a-zA-Z9-0\rbrack}, corrispondenza tra un gruppo di caratteri
    \item \code{\lbrack \^{}abc\rbrack}, non-corrispondenza tra un gruppo di caratteri
\end{itemize}

Ad esempio per trovare tutti i file che iniziano con un numero e hanno un estensione di tre caratteri:
\begin{lstlisting}[style=bash]
    utente@host:~$ find [0-9]*.???
\end{lstlisting}

Facciamo un esempio più complicato, il seguente codice trova tutti i file che hanno un nome che inizia con un numero compreso tra 10 e 20.
\begin{lstlisting}[style=bash]
    utente@host:~$ ls 1[0-9][^0-9]* 20[^0-9]*
\end{lstlisting}
Notiamo innanzitutto che il comando \textit{ls} prende due intervalli, il primo trova i file che iniziano con il carattere 1 e che hanno
come secondo carattere un numero da 0 a 9, il terzo carattere però non può essere un altro numero, in questo modo troviamo tutti i file
che iniziano con un numero compreso tra 10 e 19. Il secondo intervallo invece trova i file che iniziano con il numero 20 (il terzo carattere)
non può essere un numero.
\end{document}