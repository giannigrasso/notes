\documentclass[../main.tex]{subfiles}
\begin{document}
Una classe in java è definita come segue
\begin{lstlisting}[style=java]
    class Libro {
        String titolo;
        String nomeAutore;
        String cognomeAutore;
        int numeroPagine;
    }

    public class BibliotecaTest{
        public static void main(String[] args) {
            Libro libro1 = new Libro();
            libro1.titolo = "Viaggio intorno al mondo";
            libro1.nomeAutore = "Pippo";
            libro1.cognomeAutore = "Pluto";
            //libro1.numeroPagine = 100;

            System.out.println("Titolo: " + libro1.titolo);
        }
    }
\end{lstlisting}
\textbf{Nota:} le variabili all'interno della classe \code{Libro} vengono chiamate \underline{campi} della classe (titolo, nomeAutore, ...).

\textbf{Nota:} L'istanza di una classe è un \underline{oggetto} (libro1).

\textbf{Nota:} Se quando istanziamo un oggetto non specifichiamo il valore di un campo verrà assegnato un valore di default
(0 per gli interi, \code{null} per le stringhe, \code{false} per i booleani).

\vspace{0.5cm}
\subsection{Costruttori}
\begin{lstlisting}[style=java]
    class Libro {
        String titolo;
        String nomeAutore;
        String cognomeAutore;
        int numeroPagine;

        // COSTRUTTORE
        Libro(String titolo, String nomeAutore, String cognomeAutore, int numeroPagine) {
            this.titolo = titolo;
            this.nomeAutore = nomeAutore;
            this.cognomeAutore = cognomeAutore;
            this.numeroPagine = numeroPagine;

        }
    }

    public class BibliotecaTest{
        public static void main(String[] args) {
            Libro libro1 = new Libro("Viaggio intorno al mondo", "Pippo", "Pluto", 100);

            System.out.println("Titolo: " + libro1.titolo);
        }
    }
\end{lstlisting}

\textbf{Nota:} è possibile definire più costruttori a patto che abbiano un numero di parametri diversi o il tipo dei parametri diversi.

È anche possibile richiamare un costruttore all'interno di un altro costruttore per non duplicare linee di codice utilizzando la keyword
\code{this}.
\begin{lstlisting}[style=java]
    class Libro {
        String titolo;
        String nomeAutore;
        String cognomeAutore;
        int numeroPagine;

        // COSTRUTTORE
        Libro(String titolo, String nomeAutore, String cognomeAutore, int numeroPagine) {
            this(titolo, nomeAutore, cognomeAutore)    
        
            this.numeroPagine = numeroPagine < 0 ? 10 : numeroPagine;

        }

        // COSTRUTTORE
        Libro(String titolo, String nomeAutore, String cognomeAutore) {
            this.titolo = titolo;
            this.nomeAutore = nomeAutore;
            this.cognomeAutore = cognomeAutore;
        }
    }
\end{lstlisting}


\end{document}