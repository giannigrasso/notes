\documentclass[../main.tex]{subfiles}
\begin{document}
Java è un linguaggio di programmazione \textbf{orientato agli oggetti} progettato per essere il più possibile indipendente dalla
piattaforma di esecuzione.

\subsection{Compilazione ed esecuzione}
\begin{itemize}
    \item \textbf{JRE}, Java Runtime Environment, è un ambiente che permette di eseguire applicazioni java, contiene la
    \textit{Java Virtual Machine} (\textbf{JVM}) e le Java APIs.
    \item \textbf{JDK}, Java Development Kit, una collezione di tools per la programmazione in Java: compilatore, debugger, eccetera.
    Esso comprende una versione della JRE al suo interno
\end{itemize}


Nel codice sorgente ogni programma Java è rappresentato da \textbf{una classe}. Il nome del file \textbf{deve} coincidere con quello
della classe.

Ecco la struttura di base di un programma Java:
\begin{lstlisting}[style=java]
    /**
    * Esempio di programma.
    */
    public class NomeClasse {
        public static void main(String[] args) {
            //codice del programma
        }
    }
\end{lstlisting}

\subsection{Alcune definizioni}
\begin{itemize}
    \item \textbf{Letterali}, un letterale è la rappresentazione di un qualsiasi valore di tipo primitivo, stringa o \code{null}, in altre
    parole tutto ciò che potrebbe essere assegnato ad una variabile
    \item \textbf{Espressioni}, un espresione è un costrutto (porzione di codice) che quando viene elaborato assume un singolo valore.
    Essa può essere composta da variabili, operatori, letterali, separatori e invocazioni di funzioni. Il suo tipo di dato dipende dagli
    elementi che la compongono
    \item \textbf{Istruzioni}, l'istruzione è l'unità di esecuzione di Java. Ogni istruzione deve essere completata con il punto e virgola
    (\code{;}), fatta eccezione per le istruzioni di controllo di flusso del codice (\code{if}, \code{while}, ...)
\end{itemize}

\end{document}