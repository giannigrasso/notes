\documentclass[../main.tex]{subfiles}
\begin{document}
La classe \code{Math} mette a disposizione le seguenti funzioni:
\begin{itemize}
    \item \code{Math.pow(base, esponente)}, elevamento a potenza
    \item \code{Math.sqrt(numero)}, radice quadrata
    \item \code{Math.log(numero)}, logaritmo naturale
    \item \code{Math.log10(numero)}, logaritmo in base 10
    \item \code{Math.sin(radianti)}, seno di un angolo in radianti
    \item \code{Math.cos(radianti)}, coseno di un angolo in radianti
    \item \code{Math.tan(radianti)}, tangente di un angolo in radianti
    \item \code{Math.random()}, genera un numero casuale tra 0.0 (compreso) e 1.0 (escluso)
\end{itemize}

E le seguenti costanti:
\begin{itemize}
    \item \code{Math.E}, base del logaritmo naturale
    \item \code{Math.PI}, pi greco
\end{itemize}

\subsection{Generare numeri casuali}
Per generare numeri casuali con \code{Math.random()} si utilizza:
\begin{lstlisting}[style=java]
    int randomInt = (int) ((max - min) * Math.random() + min)
\end{lstlisting}
Dove \code{max} rappresenta il valore massimo del range (non compreso) e min quello minimo (compreso).

Se ad esempio volessi avere un range contenente tutti i numeri da 10 a 50 (entrambi compresi) sarebbe:
\begin{lstlisting}[style=java]
    int randomInt = (int) ((51 - 10) * Math.random() + 10) //il 51 non è compreso
\end{lstlisting}


\end{document}