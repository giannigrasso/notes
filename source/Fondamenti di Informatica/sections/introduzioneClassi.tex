\documentclass[../main.tex]{subfiles}
\begin{document}
\subsection{La classe \code{Scanner}}
La classe \code{Scanner} è un semplice scanner di testo che può analizzare tipi primitivi e stringhe.
Per utilizzare uno scanner è necessario importare la classe, istanziarlo e \textbf{chiuderlo}:
\begin{lstlisting}[style=java]
    import java.util.Scanner; //importa la classe

    public class EsempioHasNextInt {
        public static void main(String[] args) {
            Scanner scanner = new Scanner(System.in); //istanzia uno scanner

            System.out.print("Inserisci un numero intero: ");
            String x = scanner.next(); //utilizza lo scanner

            scanner.close(); //chiude lo scanner
        }
    }

\end{lstlisting}

\subsubsection{Anomalia dell'input}
Quando utilizziamo uno scanner per leggere un numero, premendo invio lasciamo \code{\textbackslash n} nel buffer, in questo modo  quando
poi proviamo a leggere una stringa, verrà letto soltanto il valore lasciato nel buffer. Per evitare questo ogni volta che leggiamo
un valore numerico con uno scanner, chiamiamo successivamente il metodo \code{nextLine()}, in modo che svuoti il buffer prima della
lettura della stringa.
\begin{lstlisting}[style=java]
    Scanner scanner = new Scanner(System.in);

    // Lettura di un numero intero
    System.out.print("Inserisci un numero intero: ");
    int numero = scanner.nextInt(); // Legge il numero, lascia '\n' nel buffer

    // Dopo aver letto un numero con nextInt(), rimane il carattere '\n' nel buffer.
    // Il metodo nextLine() viene utilizzato qui per consumare quel '\n' ed evitare
    // problemi nella successiva lettura di una stringa.
    scanner.nextLine();

    // Lettura di una stringa
    System.out.print("Inserisci una stringa: ");
    String stringa = scanner.nextLine();
\end{lstlisting}

Per verificare che l'utente inserisca un valore del tipo aspettato quando usiamo uno scanner possiamo usare:
\begin{lstlisting}[style=java]
    System.out.print("Inserisci un numero intero: ");

    // Ciclo che continua fino a quando non viene inserito un numero intero valido
    while (!scanner.hasNextInt()) { // hasNextInt() controlla se il prossimo input
                                    // è un numero intero, ma non consuma il valore
        System.out.print("Input non valido. Inserisci un numero intero: ");
        scanner.nextLine(); // Scarta l'input non valido
    }

    // Una volta ottenuto un intero, lo memorizziamo e lo stampiamo
    int number = scanner.nextInt();
\end{lstlisting}

\pagebreak
\subsection{Stringhe}
\subsubsection{Confronti tra stringhe}
\begin{lstlisting}[style=java]
    boolean uguali = str1.equals(str2);
\end{lstlisting}

\subsubsection{Conversione di tipi primitivi}
\begin{itemize}
    \item Da \code{String}  a \code{int} \begin{lstlisting}[style=java]
        String str = "10";
        int num = Integer.parseInt(str);
    \end{lstlisting}
    \item Da \code{String}  a \code{double} \begin{lstlisting}[style=java]
        String str = "17.42e-2";
        double num = Double.parseDouble(str);
    \end{lstlisting}
    \item Da valore numerico a \code{double} \begin{lstlisting}[style=java]
        int i = 42;
        String s1 = "" + i;
        String s2 = String.valueOf(i);
        String s3 = Integer.toString(i);
    \end{lstlisting}
    \item Da \code{String} a valore numerico \begin{lstlisting}[style=java]
        int i1 = Integer.parseInt("17030075");
        int i2 = Integer.valueOf("17030075");
    \end{lstlisting}
\end{itemize}

\subsubsection{Operazioni sulle stringhe}
\begin{itemize}
    \item \code{s1.length()}, lunghezza della stringa
    \item \code{s1.charAt(i)}, carattere alla posizione \code{i}
    \item \code{s1.substring(i, j)}, nuova stringa formata dai caratteri da posizione \code{i} inclusa fino a \code{j} esclusivamente
    \item \code{s1.indexOf(ch)}, ottieni l'indice del carattere o della stringa all'interno di \code{s1}
    \item \code{s1.toLowerCase()}, ottieni una copia tutto in minuscolo
    \item \code{s1.toUpperCase()}, ottieni una copia tutto in maiuscolo
    \item \code{s1.equalsIgnoreCase(s2)}, confronto ignorando maiuscole e minuscole
    \item \code{s1.trim()}, ottieni una copia senza spazi a inizio  e fine
\end{itemize}

\pagebreak
\subsection{La classe \code{Math}}
La classe \code{Math} mette a disposizione le seguenti funzioni:
\begin{itemize}
    \item \code{Math.pow(base, esponente)}, elevamento a potenza
    \item \code{Math.sqrt(numero)}, radice quadrata
    \item \code{Math.log(numero)}, logaritmo naturale
    \item \code{Math.log10(numero)}, logaritmo in base 10
    \item \code{Math.sin(radianti)}, seno di un angolo in radianti
    \item \code{Math.cos(radianti)}, coseno di un angolo in radianti
    \item \code{Math.tan(radianti)}, tangente di un angolo in radianti
    \item \code{Math.random()}, genera un numero casuale tra 0.0 (compreso) e 1.0 (escluso)
\end{itemize}

E le seguenti costanti:
\begin{itemize}
    \item \code{Math.E}, base del logaritmo naturale
    \item \code{Math.PI}, pi greco
\end{itemize}

\subsection{Generare numeri casuali}
Per generare numeri casuali con \code{Math.random()} si utilizza:
\begin{lstlisting}[style=java]
    int randomInt = (int) ((max - min) * Math.random() + min)
\end{lstlisting}
Dove \code{max} rappresenta il valore massimo del range (non compreso) e min quello minimo (compreso).

Se ad esempio volessi avere un range contenente tutti i numeri da 10 a 50 (entrambi compresi) sarebbe:
\begin{lstlisting}[style=java]
    int randomInt = (int) ((51 - 10) * Math.random() + 10) //il 51 non è compreso
\end{lstlisting}


\end{document}