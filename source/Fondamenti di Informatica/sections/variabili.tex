\documentclass[../main.tex]{subfiles}
\begin{document}
Ad ogni variabile è associata una zona di memoria. Le variabili vanno dichiarate fornendo il tipo di dato e il nome.
Una variabile è accessibile dal momento in cui viene dichiarata \textbf{fino alla fine del blocco in cui è contenuta}. Ecco un esempio:
\begin{lstlisting}[style=java]
    int nomeVariabile; //non inizializzata
    int x = 20 + 30; //inizializzata

    int y = 1, z; //dichiara e assegna y e dichiara z

    double d = z / (y * 1.75); //svloge prima l'espressione e poi l'assegnazione
\end{lstlisting}
\vspace{0.5cm}
\textbf{Nota:} prima di poter leggere il valore di una variabile, essa deve essere inizializzata, in caso contrario il codice non compila.
\begin{lstlisting}[style=java]
    int b;
    if (condizione) {
        b = 0;
    }
    System.out.println("b = " + b); //non compila. b potrebbe non essere inizializzata
\end{lstlisting}

\end{document}