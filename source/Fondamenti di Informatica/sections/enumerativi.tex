\documentclass[../main.tex]{subfiles}
\begin{document}
Un enumerativo definisce un nuovo tipo di dato. Se ad esempio vogliamo creare un tipo \code{Semaforo} con i soli valori \code{ROSSO, GIALLO, VERDE}
potremmo usare un \code{enum} e utilizzarlo nel seguente modo:
\begin{lstlisting}[style=java]
    enum Semaforo {
        ROSSO, GIALLO, VERDE
    }

    public class Prova {
        public static void main(String[] Args) {
            Semaforo statoSemaforo = Semaforo.ROSSO;
            Semaforo statoSemaforo = Semaforo.GIALLO;
            Semaforo statoSemaforo = Semaforo.VERDE;
        }
    }
\end{lstlisting}
\textbf{Nota:} gli \code{enum} si creano fuori dalla classe.

\textbf{Nota:} il valore di un tipo \code{enum} è costante, non si può cambiare il contenuto delle graffe fuori dalla classe.

\subsection{Operazioni}
\begin{itemize}
    \item \code{v1.ordinal()} permette di scoprire la posizione del valore \code{v1} nella lista dei valori di tipo enumerativo (si parte da 0)
    \begin{lstlisting}[style=java]
        enum Stagione {
            PRIMAVERA, ESTATE, INVERNO, AUTUNNO
        }
        Stagione stagioneFredda = Stagione.INVERNO;
        System.out.println(stagioneFredda.ordinal()); //Stampa 3
    \end{lstlisting}

    \item \code{NomeTipoEnum.valueOf(s1)} converte la stringa \code{s1} nel valore corrispondente all'enumerativo \code{NomeTipoEnum}
    \begin{lstlisting}[style=java]
        enum Stagione {
            PRIMAVERA, ESTATE, INVERNO, AUTUNNO
        }
        Stagione inverno = Stagione.valueOf("INVERNO");
    \end{lstlisting}

    \item \code{NomeTipoEnum.values()} restituisce un array contenente tutti i valori dell'enumerativo \code{NomeTipoEnum}
    \item \begin{lstlisting}[style=java]
        enum Stagione {
            PRIMAVERA, ESTATE, INVERNO, AUTUNNO
        }
        for (Stagione stagione : Stagione.values()) {
            System.out.print(stagione + " è la stagione numero ");
            System.out.println(giorno.ordinal() + 1);
        }
    \end{lstlisting}
\end{itemize}

\pagebreak
\subsection{Enum e switch}
\begin{lstlisting}[style=java]
    enum Stagione {
        PRIMAVERA, ESTATE, AUTUNNO, INVERNO
    }

    public class Prova {
        public static void main(String[] Args) {
            Stagione stagione = Stagione.valueOf(input.nextLine().toUpperCase());
            switch(stagione) {
                case PRIMAVERA
                    ...
                    break
                case ESTATE
                    ...
                    break
                case AUTUNNO
                    ...
                    break
                case INVERNO
                    ...
                    break
            }
        }
    }
\end{lstlisting}

\vspace{0.75cm}
\textbf{Note:} Attenzione, se si utilizza la sintassi nuova, per poter compilare il codice bisogna avere inserito \underline{tutti i casi}
oppure definire un valore di default.
\begin{lstlisting}[style=java]
    enum Stagione {
        PRIMAVERA, ESTATE, AUTUNNO, INVERNO
    }

    public class Prova {
        public static void main(String[] Args) {
            String mesi = switch (stagione) {
                case PRIMAVERA -> "Marzo, Aprile, Maggio";
                case ESTATE -> "Giugno, Luglio, Agosto";
                case AUTUNNO -> "Settembre, Ottobre, Novembre";

            }
        }
    }
\end{lstlisting}
Questo esempio non compila perchè \underline{manca il caso INVERNO}.


\end{document}