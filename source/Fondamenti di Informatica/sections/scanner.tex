\documentclass[../main.tex]{subfiles}
\begin{document}
La classe /code{Scanner} è un semplice scanner di testo che può analizzare tipi primitivi e stringhe.
Per utilizzare uno scanner è necessario importare la classe, istanziarlo e \textbf{chiuderlo}:
\begin{lstlisting}[style=java]
    import java.util.Scanner; //importa la classe

    public class EsempioHasNextInt {
        public static void main(String[] args) {
            Scanner scanner = new Scanner(System.in); //istanzia uno scanner

            System.out.print("Inserisci un numero intero: ");
            String x = scanner.next(); //utilizza lo scanner

            scanner.close(); //chiude lo scanner
        }
    }

\end{lstlisting}

Quando utilizziamo uno scanner per leggere un numero, premendo invio lasciamo \code{\textbackslash n} nel buffer, in questo modo  quando
poi proviamo a leggere una stringa, verrà letto soltanto il valore lasciato nel buffer. Per evitare questo ogni volta che leggiamo
un valore numerico con uno scanner, chiamiamo successivamente il metodo \code{nextLine()}, in modo che svuoti il buffer prima della
lettura della stringa.
\begin{lstlisting}[style=java]
    Scanner scanner = new Scanner(System.in);

    // Lettura di un numero intero
    System.out.print("Inserisci un numero intero: ");
    int numero = scanner.nextInt(); // Legge il numero, lascia '\n' nel buffer

    // Dopo aver letto un numero con nextInt(), rimane il carattere '\n' nel buffer.
    // Il metodo nextLine() viene utilizzato qui per consumare quel '\n' ed evitare
    // problemi nella successiva lettura di una stringa.
    scanner.nextLine();

    // Lettura di una stringa
    System.out.print("Inserisci una stringa: ");
    String stringa = scanner.nextLine();
\end{lstlisting}

Per verificare che l'utente inserisca un valore del tipo aspettato quando usiamo uno scanner possiamo usare:
\begin{lstlisting}[style=java]
    System.out.print("Inserisci un numero intero: ");

    // Ciclo che continua fino a quando non viene inserito un numero intero valido
    while (!scanner.hasNextInt()) { // hasNextInt() controlla se il prossimo input
                                    // è un numero intero, ma non consuma il valore
        System.out.print("Input non valido. Inserisci un numero intero: ");
        scanner.nextLine(); // Scarta l'input non valido
    }

    // Una volta ottenuto un intero, lo memorizziamo e lo stampiamo
    int number = scanner.nextInt();
\end{lstlisting}




\end{document}