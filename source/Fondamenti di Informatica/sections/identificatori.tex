\documentclass[../main.tex]{subfiles}
\begin{document}
Gli identificatori sono i nomi delle variabili, delle procedure e delle classi. Un identificatore è una sequenza \textbf{senza spazi} e
di lunghezza illimitata di lettere, cifre e simboli \textbf{\_} e \textbf{\$}. Inoltre un identificatore non può essere identico
ad una \textit{keyword}, ad un valore booleano (\code{true} o \code{false}) o al valore \code{null}.

\begin{itemize}
    \item \textbf{nomi delle classi}, si utilizza la sintassi \textit{UpperCamelCase}.
    \item\textbf{nomi delle variabili}, si utilizza la sintassi \textit{lowerCamelCase}.

\end{itemize}

\textbf{Nota:} Java è case sensitive.

\end{document}