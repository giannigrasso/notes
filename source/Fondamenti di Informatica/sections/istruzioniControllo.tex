\documentclass[../main.tex]{subfiles}
\begin{document}
\subsection{Switch}
Ci sono due sintassi per l'istruzione \code{switch} in Java:
\begin{itemize}
    \item \begin{lstlisting}[style=java]
        switch (espressione) {
            case valore1:
                ...;
                break;
            case valore2:
                ...;
                break;
            case valore3:
                ...;
                break;
            default:
                ...;
        }
    \end{lstlisting}
    \textbf{Nota:} L'opzione \code{default} è opzionale. Se non è presente il \code{break}, il programma continuerà al \code{case} sucessivo,
    \underline{anche se la condizione non è soddisfatta}.

    È anche possibile concatenare più \code{case} su una sola linea se la condizione è la stessa:
    \begin{lstlisting}[style=java]
        switch (espressione) {
            case valore4: case valore6: case valore9:
                ...;
                break;
            case valore2:
                ...;
                break;
            default:
                ...;
        }
    \end{lstlisting}

    \item \begin{lstlisting}[style=java]
        switch (espressione) {
            case valore1 -> {
                ...;
            }
            case valore2 -> {
                ...;
            }
            case default -> {
                ...;
            }
        }
    \end{lstlisting}
    \textbf{Nota:} questa versione non necessita dei \code{break}, impedisce il \code{fall-trough}

    \vspace{0.5cm}
    Analogamente all'esempio di prima possiamo fare:
    \begin{lstlisting}[style=java]
        switch (espressione) {
            case valore1, valore2, valore3 -> ...;
            case valore4, valore5, valore6 -> ...;
            default -> ...;
        }
    \end{lstlisting}
    \textbf{Nota:} per assegnare un valore quando si utilizzano i blocchi di codice dobbiamo usare la keyword \code{yield}.
    Se vogliamo assegnare un valore di default dobbiamo utilizzarlo per forza.
\end{itemize}

\pagebreak
Gli \code{switch} funzionano con:
\begin{itemize}
    \item \code{char}
    \item \code{byte}
    \item \code{short}
    \item \code{int}
    \item \code{String}
\end{itemize}
\textbf{Nota:} non funziona con valori decimali come ad esempio \code{double}.

\vspace{1cm}
\subsection{Operatore terniario}
Detto anceh \code{inline if}, restituisce il primo valore se la condizione è vera mentre restituisce il secondo se la condizione è falsa:
\begin{lstlisting}[style=java]
    condizione ? valoreSeVero : valoreSeFalso
\end{lstlisting}

Ad esempio:
\begin{lstlisting}[style=java]
    if (n % 2 == 0) {
        prossimo = n / 2;
    } else {
        prossimo = 3 * n + 1;
    }

    // sono uguali
    prossimo = (n % 2 == 0) ? (n / 2) : (3 * n + 1);
\end{lstlisting}

\vspace{1cm}
\subsection{Ciclo \code{for}}
\begin{lstlisting}[style=java]
    for (inizializzazione; condizione; aggiornamento) {
        seqIstruzioni;
    }
\end{lstlisting}
\textbf{Nota:} nessuno dei parametri del ciclo \code{for} (\code{inizializzazione; condizione; aggiornamento}) è obbligatorio.

È anche possibile utilizzare più variabili o condizioni:
\begin{lstlisting}[style=java]
    for (double i = 0.5, j = 39; i + j < 40; i++, j--) {
        ...
    }
\end{lstlisting}

\vspace{1cm}
\subsection{L'istruzione \code{continue} e \code{break}}
Queste due istruzioni vengono utilizzate all'interno di un ciclo:
\begin{itemize}
    \item \code{continue} fa continuare il ciclo dall'iterazione sucessiva
    \item \code{break} esce dal ciclo corrente
\end{itemize}

\subsection{L'istruzione \code{do while}}
Prima esegue le istruzioni (almeno una volta) poi verifica se la condizione è vera.
\begin{lstlisting}[style=java]
    do {
        seqIstruzioni;
    } while (condizione);
\end{lstlisting}


\end{document}